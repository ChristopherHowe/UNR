\documentclass{assignment-x}

\hmwkClass{CS 446}
\hmwkTitle{Prgramming Assignment 2, threading}
\hmwkDueDate{March 2, 2024}
\hmwkClassTime{Section 1001}
\hmwkClassInstructor{Sarah Davis}
\hmwkAuthorName{Christopher Howe}

\begin{document}
\maketitle
\pagebreak

\section{Question 1}
\subsection{Instructions}
Run the looped sum and threaded sum several times with a data file of your choice.

\subsection{What output is the same when a program with the same data is run many times? Explain.}
The total value of the array remains constant every time the program is executed. This is expected since the values within the file do not change between runs.

\subsection{What output changes when a program is run with the same data many times? Explain.}
The output indicating the time required for the program to run changes with each execution. The non-deterministic nature of how processes and threads alternate their time on the CPU results in varying execution times.

\section{Question 2}
\subsection{Instructions}
Run looped\_sum with the tenValues.txt file multiple times.

\subsection{Do you get the same sum when you run threaded\_sum with ten\_values.txt and no lock?}
The threaded\_sum executable without a lock consistently produces the correct sum. However, in scenarios with concurrent writes to a critical region, especially those involving reads, collisions and race conditions can occur. For instance, one thread reading the current totalSum value while another thread attempts to increment it can lead to unpredictable outcomes.

\subsection{Do you get the same sum when you run threaded\_sum with ten\_values.txt and a lock?}
Yes, running the threaded\_sum executable with a lock and 10 values consistently produces the correct sum. This aligns with the expected behavior, as using a lock to access totalSum is a good practice to prevent race conditions and collisions between threads.

\subsection{How does the run time of looped\_sum and threaded\_sum (locked AND not locked) compare}
For this input size, looped\_sum outperforms threaded\_sum significantly. The threaded approach introduces overhead, which is unnecessary for calculating the sum of only 10 values. The average runtime for the looped approach is less than a microsecond, while the threaded approach takes an average of 256 microseconds. The locked and non-locked threading approaches perform similarly. Refer to the attached table for detailed results.

\subsection{Is the total time to calculate the sum for the three cases different? Were they what you expected? Why or why not?}
The total time to calculate the sum for the three cases differs. I did not expect the threaded approach to perform significantly worse, but it makes sense when considering the context switching required for each thread.

\section{Question 3}
\subsection{Instructions}
Run looped\_sum with the oneThousandValues.txt file multiple times. If runtimes do not vary consistently, try running with oneHundredMillion.txt instead.

\subsection{Do you get the same sum when threaded\_sum runs with oneThousandValues.txt and no lock?}
The threaded\_sum executable without a lock consistently produces the correct sum with 1000 values. However, there is a potential for inconsistency due to concurrent writes to a critical region, as described earlier.

\subsection{Do you get the same sum when threaded\_sum runs with oneThousandValues.txt and a lock?}
Yes, running the threaded\_sum executable with a lock and 1000 values consistently produces the correct sum, aligning with expected behavior.

\subsection{How does the run time (in ms) of looped\_sum and threaded\_sum (locked AND not locked) compare?}
After running this test 10000 times, the performance advantage of the looped sum remains apparent at a thousand values. The looped sum averages 0.002 ms, while the threaded approach takes 0.479 ms and 0.464 ms. Threading does not offer improved performance at this level.

\subsection{Is the total time to calculate the sum for the three programs different? Were they what you expected? Why or why not?}
Yes, there is a noticeable difference in the runtimes of these programs. However, the performance of the locking and non-locking threaded approaches remains similar. This is expected, given that the number of locking and unlocking operations is proportional to the number of threads, making little difference in comparison to the 1000+ operations required to add the values.

\section{Question 4}
\subsection{Does the use of a lock in a threaded program have any impact on performance? How does the number of threads and the amount of data affect the performance of the threaded program with and without locks?}
The use of a lock in a threaded program significantly impacts performance. When experimenting with locking and unlocking for every read from the array, there was a considerable increase in runtime—from 121 ms to 6000 ms—when using an input file with 100,000,000 values and the same thread count.

The number of threads also affects program performance. There appears to be an optimal number of threads for optimal performance, and for the test machine, that number is 12. This aligns with the machine having 6 CPUs and 12 virtual cores.

Additionally, the amount of data influences the performance of the threaded program. Larger file sizes increase the required time for the threaded program to execute, although this increase is not linear. Jumps from 1000 to 100000 values only correspond to a 0.1 ms (25\%) increase in runtime. This is not the case for the looped approach.

\section{Question 5}
\subsection{Is the lock necessary to compute consistent values every time the program is run? Why or why not? Why do you think that occurs? You should run the program with and without a lock and with a few different data files to get the full picture.}
The lock is necessary to compute consistent values every time the program is run. While some unsafe concurrent programs may yield correct results occasionally, they are not reliable. Locks are implemented to ensure that concurrent programs consistently perform as expected. In this program, the lock should be included to protect the totalSum variable. While not every implementation will lead to a race condition, some may.

\section{Question 6}
\subsection{What considerations decided what was the Critical Section? Explain.}
The critical section for this program is the totalSum variable. The critical section of any program is any resource that multiple threads/processes have access to simultaneously. Although all threads have access to the data section where the int arrays are stored, they do not access the same portion of that array. Therefore, the integer array is not a critical region. If desired, the program could be written to have separate arrays for each thread. The totalSum variable is a critical region since multiple threads can read and write to it simultaneously.

\pagebreak

\begin{table}[htbp]
    \centering
    \begin{tabular}{|l|l|l|l|l|}
        \hline
        \textbf{File} & \textbf{Operation} & \textbf{Repeats} & \textbf{Time (no locking)} & \textbf{Time (with locking)} \\
        \hline
        ten.txt & threaded & 10000 & 0.257 ms & 0.256 ms \\
        ten.txt & looped & 10000 & 0.000 ms & \\
        oneThousand.txt & threaded & 10000 & 0.479 ms & 0.464 ms \\
        oneThousand.txt & looped & 10000 & 0.002 ms & \\
        100000-random-numbers.txt & threaded & 1000 & 0.553 ms & 0.538 ms \\
        100000-random-numbers.txt & looped & 1000 & 0.215 ms & \\
        10000000-random-numbers.txt & threaded & 100 & 5.447 ms & 5.739 ms \\
        10000000-random-numbers.txt & looped & 50 & 20.426 ms & \\
        \hline
    \end{tabular}
    \caption{Results}
    \label{tab:my_table}
\end{table}

\end{document}
