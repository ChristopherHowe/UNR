\documentclass{assignment-x}

\hmwkClass{CPE 400}
\hmwkTitle{Project Progress}
\hmwkDueDate{March 17, 2024}
\hmwkClassTime{Section 1001}
\hmwkClassInstructor{Igor Remizov}
\hmwkAuthorName{Christopher Howe}

\begin{document}
\maketitle
\pagebreak

\section{What is the project you decided to proceed with? Provide brief description of that project.}
The project I have decided to proceed with was the javascript simulation of various networking protocols. After some thought, I found that modifying docker containers to use AI to improve network preformance would be an enormous project, something that could be researched for years. 

The project I have settled on is creating a simulation of networking protocols to make it easier to learn about them in a hands on way. The protocols I am planning to demonstrate are IP, HTTP, and TCP. If I end up having more time, I may also implement the DNS protocol. The project uses reactflow to create a graph of nodes and edges that can be moved around, panned, etc. The project will allow users to add hosts, routers, and switches. The project will show packets traversing the nodes and show how they are received and processed by hosts. 

\section{What systems/tools/libraries you decided to use to complete this project?}
I am currently using a next js and react framework. I am using the reactflow library to handle adding the nodes and such. I may also look for libraries and tools to model some of the protocols so I don't have to implement them all from scratch. I also have not settled on any tools to add delays and latency to the simulation to make it more realistic.

\section{Based on the project you have selected, what results are you looking to accomplish?}
I am looking to have a visual demonstration of TCP, UDP, HTTP, and IP protocols. I will have created a website that allows users to create nodes and connections in a network to simulate this functionality. The simulation should be able to have delays as well in order to measure the impacts of different networking techniques. 

\section{Progress}
Currently, I have the entire web framework set up with a nice looking UI. I have the ability to add new nodes to the graph with a random IP address and mac address. Code exists to make sure that the mac addresses are unique. I've attached screenshots of the progress I have made so far.
\img{dialog}{A Dialog used to add new hosts}
\img{node}{The current graph is able to add new nodes with a random IP address and random mac address}
\end{document}
